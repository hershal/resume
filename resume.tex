\documentclass[margin]{res}
\setlength{\textwidth}{5.8in} % set width of text portion
\usepackage{enumitem}
\usepackage{multicol}
\usepackage{changepage}
\usepackage[none]{hyphenat}
\usepackage{hyperref}
\usepackage{soul}
\setul{1pt}{.4pt}

\begin{document}
% Center the name over the entire width of resume:
% Draw a horizontal line the whole width of resume:
% \moveleft\hoffset\vbox{\hrule width\resumewidth height 1pt}
% \smallskip
% \moveleft.5\hoffset\centerline{}
% \moveleft.5\hoffset\centerline{}
% \moveleft.5\hoffset\centerline{\large\bf Hershal Bhave}
% \moveleft.5\hoffset\centerline{817-915-1303}
\vspace{-3.5em}
\name{Hershal Bhave}
\phone{817-915-1303}
\address{6109 Shadow Valley Dr Unit A\\
  Austin, TX 78731}
\address{\href{mailto:hershal.bhave@gmail.com}{\ul{\texttt{hershal.bhave@gmail.com}}} \\
  \href{https://github.com/hershal}{\ul{\texttt{github.com/hershal}}}}
\begin{resume}
  \section{OBJECTIVE}
  To use cutting-edge technology to solve challenging software engineering and
  design problems.
  \section{EDUCATION}
  {\sl Bachelor of Science,} Electrical and Computer Engineering \\
  The University of Texas at Austin, May 2016 \\
  {\sl Concentration:} Computer Architecture, Embedded Systems, Software Architecture,
  Applied Mathematics

  \vspace{-.5em}
  \begin{multicols}{2}
    \begin{itemize}
    \item Mathematical Modeling \& Analysis
    \item Concurrent \& Distributed Systems
    \item Real-Time Operating Systems
    \item Computer Architecture
    \item Digital System Design (Verilog)
    \item Numerical Analysis
    \item Scientific Computation
    \item Embedded Systems I and II
    \item Software Development I and II
    \item Signals \& Systems
    \item Algorithms (Graph Theory)
    \item Probability \& Random Processes
    \item Software Engineering \& Design
    \end{itemize}
  \end{multicols}

  \section{INTERESTING PROJECTS}
  % {\sl M374M} \hfill \href{https://github.com/hershal/m374m}
  % {\ul{\texttt{github.com/hershal/m374m}}} \\
  % 100\% automated build system for homework, notes, and study sheets written in
  % \LaTeX{}. Uses tikz, pgf, and Octave for graphics; CMake for build automation;
  % and Travis CI for continuous integration.
  % \vspace{0.5em} \\
  In my free time, I explore cutting-edge technologies and create tools that
  help me stay in a state of flow. My personal projects are on my
  \href{https://github.com/hershal}{\ul{\sl{github}}}, written in languages such
  as JavaScript, C++, Python, and Emacs Lisp.
  \vspace{0.5em}\\
  {\sl QuARC} \hfill
  \href{https://www.youtube.com/watch?v=MEig9XBwUmU}{\sl \ul{YouTube 1}}\quad
  \href{https://www.youtube.com/watch?v=ggipd0oqxwY}{\sl \ul{YouTube 2}}\quad
  \href{https://github.com/r2labs}{\ul{\texttt{github.com/r2labs}}} \\
  Senior Design project; utilizes a homebrew Real-Time OS
  (\href{https://github.com/r2labs/uncleos}{\sl \ul{UncleOS}}) running on a TI
  TM4C and a custom \href{http://www.ros.org}{\sl \ul{Robot OS}}
  \href{https://github.com/r2labs/quarc-inverse-kinematics}{\sl \ul{inverse
      kinematics node}} running on a Qualcomm IFC6410 for precise
  \href{https://github.com/r2labs/quarc-user-interface}{\sl \ul{manipulation}}
  of a Lynxmotion AL5D 4-degrees-of-freedom robotic arm. Utilizes
  \href{http://opencv.org}{\sl\ul{OpenCV}} to
  \href{https://github.com/r2labs/quarc-vision}{\sl \ul{detect and sort
      objects}} and a \href{https://github.com/r2labs/quarc-web-interface}{\sl
    \ul{web-based dashboard}} for controls.
  \vspace{0.5em}\\
  {\sl SuperTicTacToe} \hfill
  \href{http://sttt.r2labs.us}{\ul{\texttt{sttt.r2labs.us}}}\quad
  \href{https://github.com/hershal/supertictactoe}{\ul{\texttt{github.com/hershal/supertictactoe}}} \\
  A recursive Tic-Tac-Toe game written in Java (server-side AI), vanilla
  JavaScript (client-side), and utilizes the Firebase DaaS for concurrency.

  \section{EXPERIENCE}
  {\sl NVIDIA Corporation}
  \hfill June 2016 --- Present \\
  SM Architect, Austin TX
  \vspace{0.25em}
  \begin{itemize}
  \item Owner of a GPU Microarchitectural State Checker written in C++.
  \item Co-maintainer of the low-level GPU ISA specification.
  \end{itemize}
  \vspace{-.3em}
  {\sl NVIDIA Corporation}
  \hfill (Summer Internship) June 2015 --- August 2015 \\
  SM Architecture Intern, Austin TX
  \vspace{0.25em}
  \begin{itemize}
  \item Wrote a GPU front-end decode-sequence generator in C++ and Python.
  \item Participated in GPU ISA studies and improvements.
  \end{itemize}
  \vspace{-.3em}
  {\sl Intel Corporation}
  \hfill (During School) August 2014 --- June 2015 \\
  GPU, ISP, and Display Verification Intern, Austin TX
  \vspace{0.25em}
  \begin{itemize}
  \item Developed a testbench harness for a block-functional display-controller
    model, written in C++ and SystemVerilog Direct Programming Interface
    (SV-DPI).
  \end{itemize}
  \vspace{-.3em}
  {\sl NVIDIA Corporation}
  \hfill (Summer Internship) June 2014 --- August 2014 \\
  SM Design Verification Intern, Santa Clara CA
  \vspace{0.25em}
  \begin{itemize}
  \item Expanded a C++11 test analysis tool to improve ISA coverage reporting
    by static analysis.
  \item Greatly improved visibility of test coverage for all opcodes and opcode
    modifiers, improved instruction check and filter logic, and exported data to
    an easy-to-read format.
  \end{itemize}
  % {\sl Intel Corporation}
  % \hfill (During School) August 2012 --- June 2014 \\
  % Full-Chip Verification Intern, Austin TX
  % \vspace{0.25em}
  % \begin{itemize}
  % \item Wrote scripts to analyze compiler warning messages.
  % \item Wrote various internal scripts for IP versioning and bug
  %   management.
  % \end{itemize}
  \section{COMPUTER SKILLS}
  {\sl Languages \& Software} \\
  \vspace{-1.0em}
  \begin{adjustwidth}{1.5em}{0pt}
    Bash, C++, git, GNU Octave, JavaScript (Node.js), \LaTeX, Emacs Lisp,
    Python, Swift
    \vspace{.5em}\\
    Experienced with GNU/Linux, GNU/Linux system administration, scripting,
    version control, and software development. My preferred editor is emacs.
  \end{adjustwidth}

  \section{EXTRA-\\CURRICULAR ACTIVITIES}
  {\sl President, IEEE Computer Society}
  \hfill December 2013 --- August 2015 \\
  \href{http://ieeecs.ece.utexas.edu}{\ul{\texttt{ieeecs.ece.utexas.edu}}}
  \begin{adjustwidth}{1.5em}{0pt}
    As president, I interface with companies, design competitions, develop
    lectures, and push the society and its officers to provide great services to
    the ECE student body.
   \end{adjustwidth}
\end{resume}
\end{document}
